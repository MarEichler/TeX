\documentclass[11pt, letterpaper]{article}

\usepackage[utf8]{inputenc}
\usepackage{titlesec}
\usepackage{geometry} %set margin
\usepackage{titlesec}
\usepackage{mathtools, amsmath,amsfonts,amssymb} %math tools 
\usepackage{lastpage} %to use last page 
\usepackage{enumitem} %enumerate package for (1, 2, 3, ...) or (a, b, c, ...)
\usepackage{fancyhdr,lastpage} %header info
\usepackage{fullpage} % changes the margin
\usepackage{ulem} %using fancy underlining (wave, dots, etc.)   
\usepackage{graphicx} %to add pictures 
\usepackage{subcaption} %for adding pictures 
\usepackage[dvipsnames]{xcolor} %for color names https://en.wikibooks.org/wiki/LaTeX/Colors
\usepackage{mathrsfs} %script letters
\usepackage{upgreek} %different style of greek letters 
\usepackage{wrapfig} %wrap text around graphic 
\usepackage{tikz} %use for new command: \circled{}
\usepackage{booktabs}
\usepackage{parskip} %new paragraph as skip
\usepackage{bbm} % for mathbb for numbers 
\usepackage{centernot} %so line through center of imply
\usepackage{multicol} %multi column
\usepackage{multirow} %multi rowssss 
\usepackage[mathscr]{euscript} %Euler Script Font
\usepackage[symbol]{footmisc} %symbols for footnotes 
\usepackage{array} %array and table
\usepackage{units} % for nice frac
\usepackage{contour} %fill in around text 
\usepackage{tabu} %type of tables 
\usepackage{tabularx} 
\usepackage{vwcol} %variable width column, different size columns 
\usepackage{soul} % strickthrough = \st{}
\usepackage{extsizes}
\usepackage{calc}
\usepackage{ifthen}
\usepackage{color,overpic}
\usepackage{hyperref}
\usepackage{nccmath}
\usepackage{cancel} % to cancel out lines in equaition \cancel{}
\usepackage{listings} %to put code in line 
\usepackage{paracol} %different size columns 
\usepackage{lipsum} % for dummy text
\usepackage{booktabs} 
\usepackage{longtable}


%caption in table are left-aligned or justified 
%https://github.com/haozhu233/kableExtra/issues/194
\usepackage{caption}
\DeclareCaptionLabelSeparator*{spaced}{\\[2ex]}
\captionsetup[table]{format=plain,justification=justified,
	singlelinecheck=false,labelsep=spaced,skip=0pt}



%setting indent 
%\usepackage{indentfirst} %indent first paragraph in section 
\newlength\tindent
\setlength{\tindent}{\parindent}
\setlength{\parindent}{0pt} %SENT INDENT TO ZERO 
\renewcommand{\indent}{\hspace*{\tindent}}

%creating fourth section
\setcounter{secnumdepth}{5} %5 so that it's in table of contents
\setcounter{tocdepth}{5} %5 so that its in table of contents 
\titleformat{\paragraph}
{\normalfont\normalsize\itshape}{\theparagraph}{1em}{}
\titlespacing*{\paragraph}
{0pt}{3.25ex plus 1ex minus .2ex}{1.5ex plus .2ex}

%pictures in multicolumns
\newenvironment{Figure}
{\par\medskip\noindent\minipage{\linewidth}}
{\endminipage\par\medskip}


%New commands
\newcommand{\xdash}[1][3em]{\rule[0.5ex]{#1}{0.55pt}} %dash lines
\newcommand{\pval}{\text{$p$-value}} %p-value in text
\newcommand{\then}{\stackrel{\text{then}}{\implies}} %implies arrow with 'then' text
\newcommand{\Then}{\stackrel{\text{Then}}{\implies}} %implies arrow with 'Then' text
\newcommand{\thenm}{$\stackrel{\text{then}}{\implies}$} %implies arrow with 'then' text for when out of math mode
\newcommand{\simiid}{\stackrel{\text{iid}}{\sim}} %iid distributions
\newcommand{\ssum}{\textstyle\sum} %smallsum 
\newcommand{\conprob}{\stackrel{\Ps}{\to}} % converge in probablity 
\newcommand{\conlaw}{\stackrel{\Ls}{\to}} % converge in law
\newcommand{\condist}{\stackrel{\Ds}{\to}} % converge in distribution 
\newcommand\circled[1]{\tikz[baseline=(char.base)]{\node[shape=circle,draw,inner sep=2pt] (char) {#1};}} %for chacters in circle 
\newcommand{\vs}{\text{ vs }} % vs for hypothesis testing
\newcommand{\asn}{\text{ as } n \to \infty} % as n \to \infty 
\newcommand{\astto}{\text{ as } t \to \infty} % as t \to \infty 
\newcommand{\as}{\text{ as }} % as in text, with spaces around
\newcommand{\nto}{n \to \infty} % n \to \infty 
\newcommand{\tto}{t \to \infty} % t \to \infty 
\newcommand{\cir[1]}{$\circled{#1}$} % short-hand circled
\newcommand\Ex[1]{\mathbb{E}\left[ #1 \right]} %expected value
\newcommand\Var[1]{\text{Var} \left( #1 \right)} %variance
\newcommand\Cov[1]{\text{Cov} \left( #1 \right)} %Covariance 
\newcommand\Cor[1]{\text{Corr} \left( #1 \right)} %Correlation  Cor(A,B) = Cov(A,B) / sqrt{ VarA, VarB} 
\newcommand\se[1]{\text{s.e.} \left( #1 \right)} %standard error 
\newcommand\sd[1]{\text{s.d.} \left( #1 \right)} %standard deviation 
\newcommand\cv[1]{\text{c.v.} \left( #1 \right)} %co-efficent of variance 
\newcommand{\df}{\stackrel{\text{def}}{=}} % define as... 
\newcommand{\set}{\stackrel{\text{set}}{=}} %set as ....
\newcommand{\norm}[1]{\left\lVert#1\right\rVert} %norm command
\newcommand{\abs}[1]{\left\lvert#1\right\rvert} %abs value command
\newcommand{\chk}{\textcolor{red}{\text{CHECK!}}} %check, red
\newcommand\independent{\protect\mathpalette{\protect\independenT}{\perp}} \def\independenT#1#2{\mathrel{\rlap{$#1#2$}\mkern2mu{#1#2}}} %indepdence PART 2/2
\newcommand{\simm}[1]{\stackrel{#1}{\sim}} %~ with something over it math mode
\newcommand{\simt}[1]{\stackrel{\text{#1}}{\sim}} %~ with something over it text
\newcommand{\simin}{\stackrel{\independent}{\sim}} %independently follows
\newcommand{\pj}[1]{\pmb{P}_{#1}} %projection matrix 
\newcommand{\inn}[1]{\left\langle#1\right\rangle} %inner product value command
\newcommand{\qed}{\phantom{x} \hfill  $\square$}
\newcommand{\prop}{\phantom{x} \hfill $\star$}
\newcommand{\ind}{\mathbbm{1}} %indicator function 
\newcommand{\phant}{\phantom{x}} 
\newcommand{\bars}[1]{\underline{\overline{#1}}}
\newcommand{\OR}{\text{OR}} %Odds Ratio (OR) 
\newcommand{\odds}[1]{\text{odds}_{\text{#1}}} %odds with subscript  
\newcommand{\outstanding}[1]{\textcolor{red}{\text{OUTSTANDING: }\text{#1}}} %odds with subscript 

\definecolor{litgray}{RGB}{240, 240, 240} \newcommand\code[1]{\colorbox{litgray}{\small{\texttt{{#1}}}}} %inline code


%new underline command 
\renewcommand{\ULdepth}{1.8pt}
\contourlength{0.8pt}
\newcommand{\myuline}[1]{%
	\uline{\phantom{#1}}%
	\llap{\contour{white}{#1}}%
}

%highlight
\newcommand{\highlight}[1]{%
	\colorbox{blue!10}{$\displaystyle#1$}} 


%Letters with Lines 
\newcommand{\R}{\mathbb{R}}
\newcommand{\Z}{\mathbb{Z}}
\newcommand{\Q}{\mathbb{Q}}
\newcommand{\N}{\mathbb{N}}
\newcommand{\I}{\mathbb{I}} %identity matrix 
\newcommand{\E}{\mathbb{E}}
\newcommand{\pr}{\mathbb{P}}
\newcommand{\C}{\mathbb{C}} 

%Script Letters 
\newcommand{\Us}{\mathcal{U}}
\newcommand{\Vs}{\mathcal{V}}
\newcommand{\Ws}{\mathcal{W}}
\newcommand{\Xs}{\mathcal{X}}
\newcommand{\Ns}{\mathcal{N}}
\newcommand{\Ps}{\mathcal{P}}
\newcommand{\As}{\mathcal{A}}
\newcommand{\Hs}{\mathcal{H}}
\newcommand{\Ls}{\mathcal{L}}
\newcommand{\mvn}{\mathcal{MVN}} %Multi-variate normal 
\newcommand{\Is}{\mathcal{I}} %Fisher's Information 
\newcommand{\Ds}{\mathcal{D}}
\newcommand{\Ys}{\mathcal{Y}}
\newcommand{\Ts}{\mathcal{T}}
\newcommand{\Fs}{\mathcal{F}}

%bold in math mode
\newcommand{\Ab}{\pmb{A}}
\newcommand{\ab}{\pmb{a}}
\newcommand{\bb}{\pmb{b}}
\newcommand{\Cb}{\pmb{C}}
\newcommand{\cb}{\pmb{c}}
\newcommand{\Db}{\pmb{D}}
\newcommand{\eb}{\pmb{e}}
\newcommand{\gb}{\pmb{g}}
\newcommand{\Hb}{\pmb{H}}
\newcommand{\hb}{\pmb{h}}
\newcommand{\Jb}{\pmb{J}}
\newcommand{\Lb}{\pmb{L}}
\newcommand{\Mb}{\pmb{M}}
\newcommand{\Pb}{\pmb{P}}
\newcommand{\Qb}{\pmb{Q}}
\newcommand{\tb}{\pmb{t}}
\newcommand{\Ub}{\pmb{U}}
\newcommand{\ub}{\pmb{u}}
\newcommand{\Vb}{\pmb{V}}
\newcommand{\vb}{\pmb{v}}
\newcommand{\Wb}{\pmb{W}}
\newcommand{\Xb}{\pmb{X}}
\newcommand{\xb}{\pmb{x}}
\newcommand{\Yb}{\pmb{Y}}
\newcommand{\Zb}{\pmb{Z}}
\newcommand{\zb}{\pmb{z}}
\newcommand{\yb}{\pmb{y}}


\newcommand{\betab}{\pmb{\beta}}
\newcommand{\mub}{\pmb{\mu}}
\newcommand{\omegab}{\pmb{\omega}}
\newcommand{\xib}{\pmb{\xi}}
\newcommand{\epsilonb}{\pmb{\epsilon}}
\newcommand{\etab}{\pmb{\eta}}
\newcommand{\lambdab}{\pmb{\lambda}}
\newcommand{\Sigmab}{\pmb{\Sigma}} 
\newcommand{\thetab}{\pmb{\theta}}
\newcommand{\deltab}{\pmb{\delta}}
\newcommand{\pib}{\pmb{\pi}}


\newcommand{\ob}{\pmb{0}}
\newcommand{\Pw}{\pmb{P_{\omega}}} %projection oberator 
\newcommand{\Pwo}{\pmb{P}_{\pmb{\omega}_0}} %projection oberator 
\newcommand{\w}{\pmb{\omega}} %bolded omega
\newcommand{\1}{\pmb{1}} % \vector of ones 


%text in mathmode
\newcommand{\Ht}{\text{H}}
\newcommand{\Ut}{\text{U}}
\newcommand{\BNt}{\text{BN}}
\newcommand{\Bias}{\text{Bias}}
\newcommand{\sgn}{\text{sign}} %sign 
\newcommand{\spn}{\text{span}} %span
\newcommand{\str}{\text{str}} % for stratified sampling 

%distributions 
\newcommand{\Gdist}{\text{Gamma}} % Gamma 
\newcommand{\Bern}{\text{Bernoulli}} %Bernoulli
\newcommand{\Expon}{\text{Exponential}} %Exponential
\newcommand{\Pois}{\text{Poisson}} % Poisson
\newcommand{\Unif}{\text{Uniform}} % Uniform 
\newcommand{\Bdist}{\text{Beta}} % Beta
\newcommand{\Binom}{\text{Binomial}} % Beta
\newcommand{\Multinom}{\text{Multinomial}} % Multinomial


%array strech \arraycolsep=1pt \def\arraystretch{1.4}   \begin{array}
\newcommand*\pack[1]{\colorbox{gray}{\texttt{\textcolor{white}{#1}}}}
	%to denote packages

\usepackage{listings}
%for inline code (to denote packages)

\lstset{
language=TeX,					% the language of the code
basicstyle=\footnotesize\ttfamily, 	% basic font setting\
%keywords (red)
keywordstyle=\color{Maroon},    % keyword style 
morekeywords={alph, arabic, Alph}, 			% if you want to add more keywords to the set
deletekeywords={...},           % if you want to delete keywords from the given language
%comments
commentstyle=\color{gray},    	% comment style
%line numbers 
firstnumber=1,                	% start line enumeration with line 1
stepnumber=1,					% the step between two line-numbers. If it's 1, each line will be numbered
%frame=single,	               	% adds a frame around the code             	
numbers=left,					% where to put the line-numbers; possible values are (none, left, right)
numbersep=5pt,					% how far the line-numbers are from the code
numberstyle=\tiny\color{gray}, 	% the style that is used for the line-numbers
% light blue works
classoffset=1, 					% starting new class
morekeywords={begin, end},
keywordstyle=\color{Cyan}\bfseries,
classoffset=0,
% dark blue words (pacages)
classoffset=2, 					% starting new class
morekeywords={itemize, enumerate},
keywordstyle=\color{blue},
classoffset=0,
frame=single,	                   % adds a frame around the code
}

%line spacing
\usepackage{setspace}
\setstretch{1.1}
%title spacing
\titlespacing*{\section}{0pt}{2ex plus 1ex minus .2ex}{1.5ex plus .2ex}
\titlespacing*{\subsection}{0pt}{2ex plus 1ex minus .2ex}{1ex plus .2ex}
\titlespacing*{\subsubsection}{0pt}{2ex plus 1ex minus .2ex}{1ex plus .2ex}


%PAGE STYLE
\geometry{letterpaper, left=.5in, right=.5in, top=1.1in, bottom=.5in} %paper type and margins 
\pagestyle{fancy}
\fancyhf{} % sets both header and footer to nothing
\renewcommand{\headrulewidth}{0pt}



%HEADER
\setlength{\headsep}{15pt}
%two lines for headers: 30 pt
%one line for headers: 15 pt

\lhead{LaTeX Guide for when I forget: Preamble Commands}
\rhead{Page \thepage \text{ of} \pageref{LastPage}}

% \rightmark = # Section | #.# Subsection 
%\renewcommand{\sectionmark}[1]{\gdef\currsection{\thesection \ #1}}
%\renewcommand{\subsectionmark}[1]{\markright{\currsection\ $\mid$ \thesubsection \  #1}}

% \rightmark = # Section 
\renewcommand{\sectionmark}[1]{\markright{\thesection \ #1 }}



\begin{document}

\title{LaTeX Guide for when I forget: Preamble Commands } 
%\author{Martha Eichlersmith}
%\date{} will automatically put the current date
\maketitle 
\newpage 



\def\arraystretch{1.5}
\begin{longtable}{p{.75in} p{.9in} p{.01in} p{3.55in}  p{1.45in} }
\textbf{Symbol}	&\textbf{Command}				&			&\textbf{In the Preamble}	 &
\textbf{Description}
\\ \hline 
\endhead
\caption*{$\dagger$ Denotes commands that were created to be used in math mode, but it's not required} \\
\caption*{$\ddagger$ Denotes commands that \textbf{need} to be in math mode} \\
\endfoot 

\multicolumn{5}{c}{\myuline{STATISTICAL VALUES}}
\\* 
$\Ex{X}$  	&\code{\textbackslash Ex\{X\}}&$\ddagger$		& \lstinline|\newcommand\Ex[1]{\mathbb{E}\left[ #1 \right]}| &
expected value
\\* 
$\Var{X}$  	&\code{\textbackslash Var\{X\}}&$\ddagger$		& \lstinline|\newcommand\Var[1]{\text{Var}\left( #1 \right)}| &
variance    
\\* 
$\Cov{X,Y}$  	&\code{\textbackslash Cov\{X,Y\}}&$\ddagger$& \lstinline|\newcommand\Cov[1]{\text{Cov}\left( #1 \right)}| &
Covarinace 
\\* 
$\Cor{X,Y}$  	&\code{\textbackslash Cor\{X,Y\}}&$\ddagger$& \lstinline|\newcommand\Cor[1]{\text{Corr}\left( #1 \right)}| &
Correlation %Cor(A,B) = Cov(A,B) / sqrt{ VarA, VarB} 
\\* 
$\se{X}$  	&\code{\textbackslash se\{X\}}&$\ddagger$		& \lstinline|\newcommand\se[1]{\text{s.e.}\left( #1 \right)}| &
standard error
\\* 
$\sd{X}$  	&\code{\textbackslash sd\{X\}}&$\ddagger$		& \lstinline|\newcommand\sd[1]{\text{s.d.}\left( #1 \right)}| &
standard deviation 
\\* 
$\cv{X}$  	&\code{\textbackslash cv\{X\}}&$\ddagger$		& \lstinline|\newcommand\cv[1]{\text{c.v.} \left( #1 \right)}| &
co-efficent of variance
\\* 
$\bars{X}$  &\code{\textbackslash bars\{X\}}&$\ddagger$		& \lstinline|\newcommand{\bars}[1]{\underline{\overline{#1}}}| &
upper and lower bars
\\* 
\OR  &\code{\textbackslash OR }				&$\dagger$		& \lstinline|\newcommand{\OR}{\text{OR}}| &
Odds Ratio (OR)
\\* 
$\odds{sub}$  &\code{\textbackslash odds\{sub\}}				&$\ddagger$		& \lstinline|\newcommand{\odds}[1]{\text{odds}_{\text{#1}}}| &
odds with subscript
\\ \hline

 

\multicolumn{5}{c}{\myuline{TILDE \quad $\sim$ \quad \code{\textbackslash sim}}}
\\*
$\simm{X}$  &\code{\textbackslash simm\{X\}}&$\ddagger$		& \lstinline|\newcommand{\simm}[1]{\stackrel{#1}{\sim}}| &
tilde with math on top 
\\* 
$\simt{text}$&\code{\textbackslash simt\{text\}}&$\ddagger$	& \lstinline|\newcommand{\simt}[1]{\stackrel{\text{#1}}{\sim}}| &
tilde with text on top 
\\* 
$\simin$  &\code{\textbackslash simin}&$\ddagger$		& \lstinline|\newcommand{\simin}{\stackrel{\independent}{\sim}}| &
follows independent distribution 
\\*  
$\simiid$ 		&\code{\textbackslash simiid }	&$\ddagger$	& \lstinline|\newcommand{\simiid}{\stackrel{\text{iid}}{\sim}}| &
follows iid distributions
\\ \hline

\multicolumn{5}{c}{\myuline{INFINITY \quad $\infty$ \quad \code{\textbackslash infty} \quad and \quad CONVERGENCE \quad $\to$ \quad \code{\textbackslash to}
}}
\\* 
$\asn$  		&\code{\textbackslash asn }	&$\ddagger$		& \lstinline|\newcommand{\asn}{\text{ as } n \to \infty} | &
as n to infinity 
\\*  
$\astto$  		&\code{\textbackslash astto}&$\ddagger$		& \lstinline|\newcommand{\astto}{\text{ as } t \to \infty}| &
as t to infinity 
\\* 
$\nto$  		&\code{\textbackslash nto}&$\ddagger$		& \lstinline|\newcommand{\nto}{n \to \infty}| &
n to infinity 
\\* 
$\tto$  		&\code{\textbackslash tto}&$\ddagger$		& \lstinline|\newcommand{\tto}{t \to \infty}| &
t to infinity 
\\* 
$\conprob$ 		&\code{\textbackslash conprob }	&$\ddagger$	& \lstinline|\newcommand{\conprob}{\stackrel{\Ps}{\to}}| &
converge in probability 
\\* 
$\conlaw$ 		&\code{\textbackslash conlaw }	&$\ddagger$	& \lstinline|\newcommand{\conlaw}{\stackrel{\Ls}{\to}}| &
converge in law
\\* 
$\condist$ 		&\code{\textbackslash condist }	&$\ddagger$	& \lstinline|\newcommand{\condist}{\stackrel{\Ds}{\to}}| &
converge in distribution 
\\ \hline 

 
\multicolumn{5}{c}{\myuline{MATHEMATICAL OPERATORS, OPERATIONS}}
\\*
$\inn{X,Y}$  &\code{\textbackslash inn\{X, Y\}}&$\ddagger$		& \lstinline|\newcommand{\inn}[1]{\left\langle#1\right\rangle}| &
inner product 
\\* 
$\norm{X}$  &\code{\textbackslash norm\{X\}}&$\ddagger$		& \lstinline|\newcommand{\norm}[1]{\left\lVert#1\right\rVert} | &
Norm 
\\* 
$\abs{X}$  &\code{\textbackslash abs\{X\}}&$\ddagger$		& \lstinline|\newcommand{\abs}[1]{\left\lvert#1\right\rvert}| &
Absolute Value 
\\*  
$\ssum$ 		&\code{\textbackslash ssum }	&$\ddagger$	& \lstinline|\newcommand{\ssum}{\textstyle\sum}| &
small sum 
\\* 
$\independent$ &\code{\textbackslash independent}&$\ddagger$& \lstinline|\newcommand\independent{\protect\mathpalette{\protect| 
\rotatebox[origin=c]{180}{$\Lsh$} \lstinline|\independenT}{\perp}}  \def\independenT#1#2{\mathrel{| 
\rotatebox[origin=c]{180}{$\Lsh$} \lstinline|\rlap{$#1#2$}\mkern2mu{#1#2}}}|&
Independent symbol 
\\* 
$\ind$  &\code{\textbackslash ind}&$\ddagger$		& \lstinline|\newcommand{\ind}{\mathbbm{1}}| &
indicator function
\\* 
$\pj{X}$  &\code{\textbackslash pj\{X\}}&$\ddagger$		& \lstinline|\newcommand{\pj}[1]{\pmb{P}_{#1}}| &
projection matrix
\\* 
$\Pw$  &\code{\textbackslash Pw}&$\ddagger$		& \lstinline|\newcommand{\Pw}{\pmb{P_{\omega}}}| &
projection operator for $\omegab$
\\* 
$\Pwo$  &\code{\textbackslash Pwo}&$\ddagger$		& \lstinline|\newcommand{\Pwo}{\pmb{P}_{\pmb{\omega}_0}}| &
projection operator for $\omegab_0$
\\ \hline
 


\multicolumn{5}{c}{\underline{TEXT WITH SYMBOLS IN MATH MODE}}
\\*  
$\then$ 		&\code{\textbackslash then}	&$\ddagger$	& \lstinline|\newcommand{\then}{\stackrel{\text{then}}{\implies}}| &
implies arrow with `then' text
\\*  
$\Then$ 		&\code{\textbackslash Then }	&$\ddagger$	& \lstinline|\newcommand{\then}{\stackrel{\text{Then}}{\implies}}| &
implies arrow with `Then' text
\\*  
\thenm 		&\code{\textbackslash thenm}		&			& \lstinline|\newcommand{\thenm}{$\stackrel{\text{then}}{\implies}$}| &
implies arrow with `then' text, out of math mode
\\* 
$\df$  	&\code{\textbackslash df}			&$\ddagger$		& \lstinline|\newcommand{\df}{\stackrel{\text{def}}{=}}| &
define as ...
\\*  
$\set$  &\code{\textbackslash set}			&$\ddagger$		& \lstinline|\newcommand{\set}{\stackrel{\text{set}}{=}}| &
set as ...
\\ \hline  


\multicolumn{5}{c}{\underline{TEXT IN MATH MODE}}
\\*  
\pval	 		&\code{\textbackslash pval}	&$\dagger$	& \lstinline|\newcommand{\pval}{\text{$p$-value}}| &
pval text
\\* 
\vs  		&\code{\textbackslash vs}	&$\dagger$			& \lstinline|\newcommand{\vs}{\text{ vs }}| &
text `\_vs\_' for hypothesis test,  
$H_0 \vs H_1$ \lstinline|$H_0 \vs H_1$ |
\\* 
\as  		&\code{\textbackslash as}	&$\dagger$			& \lstinline|\newcommand{\as}{\text{ as }}| &
text `\_as\_', 
$x \as y$ \lstinline|$x \as y$ |
\\*  
\BNt	 		&\code{\textbackslash BNt}	&$\dagger$	& \lstinline|\newcommand{\BNt}{\text{BN}}| &
BN (Binomial distribution short hand)
\\*  
\Bias	 		&\code{\textbackslash Bias}	&$\dagger$	& \lstinline|\newcommand{\Bias}{\text{Bias}}| &
Bias 
\\*  
\sgn	 		&\code{\textbackslash sgn}	&$\dagger$	& \lstinline|\newcommand{\sgn}{\text{sign}}| &
sign 
\\*  
\spn	 		&\code{\textbackslash spn}	&$\dagger$	& \lstinline|\newcommand{\spn}{\text{span}}| &
span 
\\*  
\str	 		&\code{\textbackslash str}	&$\dagger$	& \lstinline|\newcommand{\str}{\text{str}}| &
stratified sampling  
\\ \hline 


\multicolumn{5}{c}{\underline{TEXT DISTRIBUTION FULL NAMES IN MATH MODE}}
\\*  
\BetaD	 		&\code{\textbackslash BetaD}	&$\dagger$	& \lstinline|\newcommand{\BetaD}{\text{Beta}}| & Beta 
\\*  
\Bern	 		&\code{\textbackslash Bern}		&$\dagger$	& \lstinline|\newcommand{\Bern}{\text{Bernoulli}}| & Bernoulli
\\*  
\Binom	 		&\code{\textbackslash Binom}	&$\dagger$	& \lstinline|\newcommand{\Binom}{\text{Binomial}}| & Binomial
\\*  
\Expon	 		&\code{\textbackslash Expon}	&$\dagger$	& \lstinline|\newcommand{\Expon}{\text{Exponential}}| & Exponential
\\*  
\Gamm	 		&\code{\textbackslash Gamm}		&$\dagger$	& \lstinline|\newcommand{\Gamm}{\text{Gamma}}| & Gamma
\\*  
\Multinom 		&\code{\textbackslash Multinom}	&$\dagger$	& \lstinline|\newcommand{\Multinom}{\text{Multinomial}}| & Multinomial
\\*  
\Pois	 		&\code{\textbackslash Pois}		&$\dagger$	& \lstinline|\newcommand{\Pois}{\text{Poisson}}| & Poisson
\\*  
\Unif	 		&\code{\textbackslash Unif}		&$\dagger$	& \lstinline|\newcommand{\Unif}{\text{Uniform}}| & Uniform 
\\ \hline



\multicolumn{5}{c}{\underline{WRITING PROOFS}}
\\*   
\qed  &\code{\textbackslash qed}			&		& \lstinline|\newcommand{\qed}{\phantom{x} \hfill  $\square$}| &
hfill the line then add square, to denote end of proof
\\*   
\prop  &\code{\textbackslash prop}			&		& \lstinline|\newcommand{\prop}{\phantom{x} \hfill $\star$}| &
hfill the line then add star, to denote proposition
\\ \hline 


\multicolumn{5}{c}{\underline{LETTERS WITH LINE}}
\\* 
$\C$	&\code{\textbackslash C}	&$\ddagger$	&\lstinline|\newcommand{\C}{\mathbb{C}}|	&Complex numbers 
\\* 
$\E$	&\code{\textbackslash E}	&$\ddagger$	&\lstinline|\newcommand{\E}{\mathbb{E}}|	&Expected Value 
\\*
$\I$	&\code{\textbackslash I}	&$\ddagger$	&\lstinline|\newcommand{\I}{\mathbb{I}}|	&Identity Matrix 
\\*
$\N$	&\code{\textbackslash N}	&$\ddagger$	&\lstinline|\newcommand{\N}{\mathbb{N}}|	&Natural number, $\Z^+$ 
\\*  
$\pr$	&\code{\textbackslash pr}	&$\ddagger$	&\lstinline|\newcommand{\pr}{\mathbb{P}}|	&Probability 
\\* 
$\Q$	&\code{\textbackslash Q}	&$\ddagger$	&\lstinline|\newcommand{\Q}{\mathbb{Q}}|	&Rational Numbers 
\\* 
$\R$	&\code{\textbackslash R}	&$\ddagger$	&\lstinline|\newcommand{\R}{\mathbb{R}}|	&Real numbers 
\\* 
$\Z$	&\code{\textbackslash Z}	&$\ddagger$	&\lstinline|\newcommand{\Z}{\mathbb{Z}}|	&Integers 
\\  \hline

\multicolumn{5}{c}{\underline{BOLD LETTERS/CHARACTERS}}
\\* 
$\Xb$	&\code{\textbackslash Xb}	&$\ddagger$	&\lstinline|\newcommand{\Xb}{\pmb{X}}|	&Uppercase X
\\*
$\xb$	&\code{\textbackslash xb}	&$\ddagger$	&\lstinline|\newcommand{\xb}{\pmb{x}}|	&Lowercase x
\\* 
$\betab$	&\code{\textbackslash betab}	&$\ddagger$	&\lstinline|\newcommand{\betab}{\pmb{\beta}}|	&Lowercase beta
\\*
$\Sigmab$	&\code{\textbackslash Sigmab}	&$\ddagger$	&\lstinline|\newcommand{\Sigmab}{\pmb{\Sigma}} |	&Uppercase Sigma, variance covariance matrix
\\*
$\ob$	&\code{\textbackslash ob}	&$\ddagger$	&\lstinline|\newcommand{\ob}{\pmb{0}}|	&Bold zero, vector or matrix of zero's
\\*
$\1$	&\code{\textbackslash 1}	&$\ddagger$	&\lstinline|\newcommand{\1}{\pmb{1}}|	&Bold one, vector or matrix of one's
\\*
\multicolumn{5}{l}{\textit{Similar newcommands are used for multiple letters; bold letters used to denote vectors and/or matrices }}
\\ \hline

\multicolumn{5}{c}{\underline{SCRIPT LETTERS}}
\\* 
$\Is$	&\code{\textbackslash Is}	&$\ddagger$	&\lstinline|\newcommand{\Is}{\mathcal{I}}|	&Fisher's Information
\\*
$\Ns$	&\code{\textbackslash Ns}	&$\ddagger$	&\lstinline|\newcommand{\Ns}{\mathcal{N}}|	&Normal Distribution
\\*
\multicolumn{5}{l}{\textit{Similar newcommands are used for multiple letters; often used to denote distributions}}
\\ \hline 




\multicolumn{5}{c}{\underline{MISCELLANIOUS}}
\\* 
\xdash  		&\code{\textbackslash xdash }	&			& \lstinline|\newcommand{\xdash}[1][3em]{\rule[0.5ex]{#1}{0.55pt}}| &
dash line
\\*  
\circled{0}		&\code{\textbackslash circled\{0\}}&		& \lstinline|\newcommand\circled[1]{\tikz[baseline=(char.base)]{\node| 
\rotatebox[origin=c]{180}{$\Lsh$} \lstinline|[shape=circle,draw,inner sep=2pt] (char) {#1};}}| &
for characters in circle
\\* 
\cir[0]	&\code{\textbackslash cir[0]}&		& \lstinline|\newcommand{\cir[1]}{$\circled{#1}$}|  &
short-hand circled
\\*  
\phant 		&\code{\textbackslash phant }	&			& \lstinline|\newcommand{\phant}{\phantom{x}}| &
phantom character (sometimes need for hfill or other commands if nothing else is on the line)
\\* 
\highlight{X}	&\code{\textbackslash highlight\{X\} }	&	& \lstinline|\newcommand{\highlight}[1]{\colorbox{blue!10}|
\phantom{xxxxxx}
\rotatebox[origin=c]{180}{$\Lsh$} \lstinline|{$\displaystyle#1$}}|  &
highlight
\\* 
\chk 		&\code{\textbackslash chk }	&			& \lstinline|\newcommand{\chk}{\textcolor{red}{\text{CHECK!}}}| &
check 
\\* 
\tiny{\outstanding{x} } &\code{\textbackslash outstanding\{x\} }&&\lstinline|\newcommand{\outstanding}[1]{\textcolor{red}{\text{OUT|
\rotatebox[origin=c]{180}{$\Lsh$} \lstinline|STANDING: }\text{#1}}}| &
specify outstanding tasks
\\* 
\code{code} 	&\code{\textbackslash code \{code\} }	&& \lstinline|\definecolor{litgray}{RGB}{240, 240, 240}	\newcommand| 
\rotatebox[origin=c]{180}{$\Lsh$} \lstinline|\code[1]{\colorbox{litgray}{\small{\texttt{{#1}}}}}|
&
inline code 
\\ \hline
\end{longtable}


\code{\textbackslash newcommand*}  or \code{\textbackslash newcommand}
 
 ``Using the starred version of \code{\textbackslash newcommand*} means that the arguments of the defined command cannot contain a blank line or \code{\textbackslash par}. This makes it a lot easier to spot runaway arguments." - \href{https://tug.org/mail-archives/texhax/2005-March/003732.html}{Source}

\newpage
\section*{Renew Commands, setup commands, etc}

\subsection*{Customized Underline}
\href{https://alexwlchan.net/2017/10/latex-underlines/}{Source: https://alexwlchan.net/2017/10/latex-underlines/}
\begin{lstlisting} 
\usepackage{<contour>} %fill in around text 
\usepackage{<ulem>} %using fancy underlining (wave, dots, etc.)
	%new underline command
	\renewcommand{\ULdepth}{1.8pt}
	\contourlength{0.8pt}
	\newcommand{\myuline}[1]{
		\uline{\phantom{#1}}
		\llap{\contour{white}{#1}}
		}
\end{lstlisting}

\begin{tabular}{ll}
\underline{abcd} \quad \underline{efgh}	&\lstinline|\underline{abcd} \quad \underline{efgh}|
\\
\myuline{abcd} \quad \myuline{efgh} 	&\lstinline|\myuline{abcd} \quad \myuline{efgh}|
\end{tabular} 

\subsection*{Captions in Tables are justified (left-aligned) vs centered}
\href{
	https://github.com/haozhu233/kableExtra/issues/194
	}{Source: 
	https://github.com/haozhu233/kableExtra/issues/194
	}
\begin{lstlisting}
\usepackage{<caption>}
\DeclareCaptionLabelSeparator*{spaced}{\\[2ex]}
\captionsetup[table]{format=plain,justification=justified,
singlelinecheck=false,labelsep=spaced,skip=0pt}
\end{lstlisting} 

\subsection*{Setting Indent }
Did not save source :( 
\begin{lstlisting}
\usepackage{<indentfirst>} %indent first paragraph in section 
\newlength\tindent
\setlength{\tindent}{\parindent}
\setlength{\parindent}{0pt} %SENT INDENT TO ZERO 
\renewcommand{\indent}{\hspace*{\tindent}}
\end{lstlisting}

\subsection*{Creating Fourth Section} 
Did not save source :(
\begin{lstlisting}
\setcounter{secnumdepth}{5} %5 so that it's in table of contents
\setcounter{tocdepth}{5} %5 so that its in table of contents 
\titleformat{\paragraph}
{\normalfont\normalsize\itshape}{\theparagraph}{1em}{}
\titlespacing*{\paragraph}
{0pt}{3.25ex plus 1ex minus .2ex}{1.5ex plus .2ex}
\end{lstlisting}

\newpage 
\subsection*{Pictures in Multicolumn environment} 
\href{
	https://tex.stackexchange.com/questions/12262/multicol-and-figures
}{Source: 
	https://tex.stackexchange.com/questions/12262/multicol-and-figures
}
\begin{lstlisting} 
\documentclass[a5paper]{<article>}
\usepackage{<multicol>,<caption>}
\usepackage[demo]{<graphicx>}
\usepackage{<lipsum>}
\newenvironment{Figure}
{\par\medskip\noindent\minipage{\linewidth}}
{\endminipage\par\medskip}

\begin{"document"}

\begin{"multicols"}{2}
\lipsum[1]
\begin{"Figure"}
\centering
\includegraphics[width=\linewidth]{myimage.png}
\end{"Figure" }

\lipsum[1]
\end{"multicols"}

\end{"document"}
\end{lstlisting}


 







\end{document} 